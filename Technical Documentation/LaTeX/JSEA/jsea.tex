\documentclass[conference, 12pt]{IEEEtran}
\IEEEoverridecommandlockouts
\usepackage{cite}
\usepackage{amsmath,amssymb,amsfonts}
\usepackage[margin=0.8in, pass, paperwidth=8.5in, paperheight=11in]{geometry}
\usepackage{algorithmic}
\usepackage{graphicx}
\usepackage{textcomp}
\usepackage{adjustbox}
\usepackage{subcaption}
\usepackage{enumitem}
\usepackage{multirow}
\usepackage{bbding, enumitem}
\usepackage{authblk}
\usepackage{booktabs}
\usepackage{xcolor}
\usepackage{colortbl}
\usepackage{url}
\usepackage{times}
\usepackage{titlesec}
\usepackage{tocloft}
\usepackage{array}
\usepackage{longtable}

\renewcommand{\thesection}{\arabic{section}}
\renewcommand{\thesubsection}{\thesection.\arabic{subsection}}
\renewcommand{\thesubsubsection}{\thesubsection.\arabic{subsubsection}}

\makeatletter
\def\section{\@startsection {section}{1}{\z@}%
    {1.0ex plus 1ex minus .2ex}%
    {1.0ex plus .2ex}% Spacing after the section
    {\normalfont\normalsize\bfseries\raggedright}}
\makeatother
% Define subsubsubsection with numbering (1.1.1.1)
\titleformat{\paragraph}
  {\normalfont\normalsize\itshape}{\theparagraph}{1em}{}

\renewcommand{\theparagraph}{\thesubsubsection.\arabic{paragraph}}
\setcounter{secnumdepth}{4}  % Allow numbering for \paragraph
\setcounter{tocdepth}{4}     % Allow \paragraph to appear in ToC
\renewcommand{\arraystretch}{1.3} % Adjust row height for better spacing


\makeatletter
\def\@seccntformat#1{\csname the#1\endcsname\quad}
\makeatother

\usepackage{fancyhdr}
\pagestyle{fancy}

\fancyhf{}
\lhead{\nouppercase {\leftmark}}
\rhead{\thepage}
\fancyfoot[L]{\hspace*{-1.9cm}\raisebox{-1.3cm}{\includegraphics[width=\paperwidth,height=1.9cm]{Footer.png}}} % Footer


\begin{document}
\onecolumn

\section{Introduction}

This report provides an in-depth analysis of job safety for both electrical and
mechanical tasks. It discusses all potential hazards, the safety measures
taken, and the team members responsible for each task.

\begin{table}[h]
    \centering
    \setlength{\arrayrulewidth}{1pt} % Thicker table lines
    \setlength{\tabcolsep}{10pt} % Column spacing
    \begin{tabular}{|p{9cm}|p{7cm}|}
        \hline
        \rowcolor{gray!20} \textbf{Environmental and Personal Protective Equipment Considerations} & \textbf{Emergency Procedures} \\
        \hline
        \vspace{-0.8\baselineskip}\begin{itemize}[leftmargin=*]
            \item Safety goggles
            \item Gloves
            \item Lab coats/aprons and non-loose clothing
            \item Closed-toe shoes
            \item Proper ventilation
        \end{itemize}
        &
        \vspace{-0.8\baselineskip}\begin{itemize}[leftmargin=*]
            \item \textbf{Electrical Shock:} Immediately shut down power and seek medical assistance.
            \item \textbf{Fire (from soldering or electronics):} Use a fire extinguisher, evacuate if necessary.
            \item \textbf{Cuts/Burns:} Apply first aid, use bandages, seek medical attention if needed.
            \item \textbf{Water Leakage on Electronics:} Shut off power, remove Kamikaze from water, dry components thoroughly.
        \end{itemize} \\
        \hline
    \end{tabular}
    \caption{Safety Guidelines and Emergency Procedures}
\end{table}

\section{Job-Safety Analysis for Mechanical Tasks}

\setlength{\arrayrulewidth}{1pt} % Set thicker line width
\begin{longtable}{|>{\raggedright}p{3.5cm}|>{\raggedright}p{3.5cm}|p{1.5cm}|>{\raggedright}p{5cm}|p{3cm}|}
    \hline
    \rowcolor{gray!20} \textbf{Task} & \textbf{Hazard} & \textbf{Risk Level} & \textbf{Safety Measures} & \textbf{Responsible Team Member} \\
    \hline
    \endfirsthead
    \hline
    \rowcolor{gray!20} \textbf{Task} & \textbf{Hazard} & \textbf{Risk Level} & \textbf{Safety Measures} & \textbf{Responsible Team Member} \\
    \hline
    \endhead
    \hline
    \multicolumn{5}{r}{\textbf{(Continued on next page)}} \\
    \endfoot
    \endlastfoot
    \textbf{Transporting Kamikaze}
        & Falling, Slipping, Back Injury
        & Low
        & \vspace{-0.8\baselineskip}\begin{itemize}[leftmargin=*]
            \item Maintain a clean and obstacle-free workspace.
            \item Wear non-slip footwear to avoid slipping.
            \item Reduce the amount of weight lifted.
        \end{itemize}
        & Abdelaziz Serour - CEO \\
    \hline
    \textbf{Using sharp tools}
        & Hand injury, Eye injury
        & High
        & \vspace{-0.8\baselineskip}\begin{itemize}[leftmargin=*]
            \item Wear safety gloves and safety glasses.
            \item Use tools correctly and store them safely.
            \item Use personal protective equipment (PPE).
        \end{itemize}
        & Mechanical and Electrical Team \\
    \hline
    \textbf{Pneumatic system}
        & Exceeding the maximum pressure allowed
        & High
        & \vspace{-0.8\baselineskip}\begin{itemize}[leftmargin=*]
            \item Use a pressure relief valve set at 10 bar, the allowable pressure for the tank.
            \item Regularly inspect and maintain the system.
        \end{itemize}
        & {\raggedright
        \vspace{-0.8\baselineskip}\begin{itemize}[leftmargin=*]
                \item Basmala Ehab - Vice Mechanical CTO
                \item Ammar Adel - Mechanical Member
            \end{itemize} 
            \par}\\
    \hline
    \textbf{Drill press}
        & Hand injury, Finger injury
        & Medium
        & \vspace{-0.8\baselineskip}\begin{itemize}[leftmargin=*]
            \item Keep hands away from moving parts.
            \item Secure materials with designated clamps before drilling.
        \end{itemize}
        & Mechanical Team \\
    \textbf{Sealing Components}
        & Improper sealing, Water leakage, Pressure failure
        & High
        & \vspace{-0.8\baselineskip}\begin{itemize}[leftmargin=*]
            \item Ensure all seals are properly fitted and tightened.
            \item Perform pressure tests before deployment.
            \item Inspect seals for wear or damage regularly.
        \end{itemize}
        & Zeyad Elbahy - Vice Mechanical CTO \\
    \hline
    \textbf{Tether and cable connections}
        & Entanglement hazards
        & Low
        & \vspace{-0.8\baselineskip}\begin{itemize}[leftmargin=*]
            \item Organize and secure cables properly.
            \item Monitor Kamikaze movement to prevent entanglement.
            \item Regularly inspect tether for wear or damage.
        \end{itemize}
        & {\raggedright
        \vspace{-0.8\baselineskip}\begin{itemize}[leftmargin=*] 
            \item Tether man 
            \item \noindent All team memeber 
        \end{itemize}
        \par} \\
    \hline
    \textbf{Operating thrusters}
        & Finger damage, Thrusters blade damage
        & High
        & \vspace{-0.8\baselineskip}\begin{itemize}[leftmargin=*]
            \item Always operate with shrouds mounted on thrusters.
            \item Always keep hands away from moving thrusters.
        \end{itemize}
        & Omar Hussien - Mechanical CTO \\
    \hline
    \caption{Task Hazards and Controls for Mechanical Tasks}
\end{longtable}

\newpage
\section{Job-Safety Analysis for Electrical Tasks}
\begin{longtable}{|>{\raggedright}p{3.5cm}|>{\raggedright}p{3.5cm}|p{1.5cm}|>{\raggedright}p{5cm}|p{3cm}|}
    \hline
    \rowcolor{gray!20} \textbf{Task} & \textbf{Hazards} & \textbf{Risk Level} & \textbf{Safety Measures} & \textbf{Responsible Team Member} \\
    \hline
    \endfirsthead
    \hline
    \rowcolor{gray!20} \textbf{Task} & \textbf{Hazards} & \textbf{Risk Level} & \textbf{Safety Measures} & \textbf{Responsible Team Member} \\
    \hline
    \endhead
    \hline
    \multicolumn{5}{r}{\textbf{(Continued on next page)}} \\
    \endfoot
    \endlastfoot
    \textbf{Soldering Electronic Components}
        & Fire hazard, Electrical hazard, Chemical inhalation, Toxic material, Eye injury
        & High
        & \vspace{-0.8\baselineskip}\begin{itemize}[leftmargin=*]
            \item Turn off soldering iron when it’s not used.
            \item Remove combustible materials from work area.
            \item Know the location of fire extinguishers and first aid kits in case of emergency.
            \item Perform work in a well-ventilated area.
        \end{itemize}
        & Electrical Team \\
    \hline
    \textbf{Setting Up the Power Supply}
        & Potential overvoltage leading to equipment damage, Fire hazards due to electrical faults or short circuits
        & High
        & \vspace{-0.8\baselineskip}\begin{itemize}[leftmargin=*]
            \item Check the voltage reading that it shows nominal 48V.
            \item Use control status indicator LED.
            \item Use fuses on powerlines preventing drivers and DC-DC converters from burning out.
        \end{itemize}
        & Electrical Team \\
    \hline
    \textbf{Maintenance of Electrical Equipment}
        & Electric shock, Possibility of equipment damage if not maintained properly
        & Medium
        & \vspace{-0.8\baselineskip}\begin{itemize}[leftmargin=*]
            \item Verify all power in the control box is off and wait 5 sec.
            \item The operator must wear non-electrically conducting gloves.
        \end{itemize}
        & Ahmed Elattar – Electrical CTO \\
    \hline
    \textbf{Ensuring ROV Communication}
        & Loss of communication will result in losing control of the ROV and could destroy the ROV and objects around it, Risk of signal interference leading to communication breakdown
        & Low
        & \vspace{-0.8\baselineskip}\begin{itemize}[leftmargin=*]
            \item Implementation of reliable communication protocols within the logical range to prevent loss of control.
            \item Avoidance of excessive distances between the joystick and Kamikaze to maintain signal accuracy.
        \end{itemize}
        & {\raggedright Electrical and Software Team \par} \\
    \textbf{Management of Tether and Cable Connections}
        & Entanglement hazards, leading to interruption of operations or equipment damage, Cable damage if not handled properly
        & High
        & \vspace{-0.8\baselineskip}\begin{itemize}[leftmargin=*]
            \item Be mindful of where you place the tether and how far the ROV is from it.
            \item Regular inspection of cables for signs of wear or damage.
        \end{itemize}
        & Electrical Team \\
    \hline
    \caption{Hazards and Controls for Electrical Tasks}
\end{longtable}

\end{document}